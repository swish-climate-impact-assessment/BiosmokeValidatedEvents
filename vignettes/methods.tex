\section{Description}\label{description}

The background and purpose of the database or data collection should be
presented for readers without specialist knowledge in that area. For
this database we should cite the original paper by Johnston et al.
(2011a) as well as the two health analyses of Hospitalisation (Martin
\emph{et al.} 2013) and Mortality (Johnston \emph{et al.} 2011b).

This will be followed by a brief description of the protocol for data
collection, data curation and quality control, and what is being
reported in the article.

The user interface should be described and a discussion of the intended
uses of the database, and the benefits that are envisioned, should be
included, together with data on how it compares with similar existing
databases. A case study of the use of the database may be presented. The
planned future development of new features, if any, should be mentioned.

The findings section can be broken into subsections with short
informative headings. There is no maximum length for this section but we
encourage authors to be concise.

\section{General Protocols}\label{general-protocols}

For each location, up to 13 yr (between 1994 and 2007) of daily air
quality data measured asPMless than 10um (PM10 ) or less than 2.5 um
(PM2.5 ) in aerodynamic diameter were examined. Air pollution data were
pro- vided by government agencies in the states of Western Australia,
New South Wales, and Tasmania. Daily averages for each site were
calculated excluding days with less than 75\% of hourly measurements. In
Sydney and Perth, where data were collected from several monitoring sta-
tions, the missing daily site-specific PM10 and PM2.5 con- centrations
were imputed using available data from other proximate monitoring sites
in the network. The daily city-wide PM10 and PM2.5 concentrations were
then estimated following the protocol of the Air Pollution and Health: a
European Approach studies (Atkinson \emph{et al.} 2001).

\section{Detailed Data Collation and Validation
Methods}\label{detailed-data-collation-and-validation-methods}

\subsection{Step 1: Imputation to fill in gaps in the
time-series}\label{step-1-imputation-to-fill-in-gaps-in-the-time-series}

First a `filling-in' procedure was used to improve data completeness. It
entailed the substitution of the missing daily values with a weighted
average, using the weights of the missing sites 3-month average
proportional to the network average. The weights are calculated against
the values from the rest of the monitoring stations. The pollutant
measures from all stations providing data were then averaged to provide
single, city-wide estimates of the daily levels of the pollutants

For each city, all days in which PM10 or PM2.5 exceeded the 95th
percentile were identified over the entire time series. These extreme
values were termed `events'. A range of sources was ex- amined to
identify the cause of particulate air pollution events, including
electronic news archives, Internet searches for other reports,
government and research agencies, satellite imagery and a Dust Storms
database. Also examined were remotely sensed aerosol optical thickness
(AOT) data to provide further information about days for which the other
methods did not.

Step 1.0 Source air pollution data. Both time series observations and
spatial data regarding site locations.

Step 1.1. NSW data downloaded from an online data server. Site locations
(Lat and Long) obtained from website.

Step 1.2. WA data sent on CD from contacts at the WA Government
Department, these were hourly data as provided. Cleaned so as only days
with \textgreater{}75\% of hours are used. Licence puts restricions on
our right to provide to a third party. Therefore those observed and
imputed data are not included, only the events.

Step 1.3. Tasmanian data sent via email from contact at the Department,
these were daily data.

Step 1.4. All data combined and Quality Control checked in the PostGIS
database.

\subsection{Step 2. Spatial data for
cities.}\label{step-2.-spatial-data-for-cities.}

\subsection{Step 3. Calculate a network
average}\label{step-3.-calculate-a-network-average}

In cities where data were collected from several monitoring stations,
the missing daily site-specific PM concentrations were imputed using
available data from other proximate monitoring sites in the network. The
daily city-wide PM concentrations were then estimated following the
protocol of the Air Pollution and Health: a European Approach studies.
Atkinson et al. (2001).

Step 3.1. Prepare Data. First it was necessary to find the minimum date
that the series of continuous observations can be considered to start.
In the Australian datasets the initial observations could not be used
because the were sometimes only one day per week, only during a
particular season or of poor quality due to teething problems with
equipment and procedures. Then it was necessary to identify missing
dates. Get a list of the sites to include -- that is with more than 70\%
observed over the time period (as defined after assessing min and max
dates of period).

Step 3.2. Loop over each station individually and calculate a daily
network average of all the other non-missing sites (ie an average of all
stations except the focal station of that iteration in the loop).

Step 3.3. Calculate three monthly seasonal mean of these non-missing
stations. Calculate a three-month seasonal mean for MISSING site.
Estimate missing days at missing sites.

Step 3.4. Join all sites for city wide averages and fill any missing
days with avg of before and after.

Step 3.5 Take the average of all sites per day for city wide averages.

Step 3.6. Fill any missing days with avg of before and after (if this is
less than 5\% of days).

\subsection{Step 4. Validate events and identify the
causes}\label{step-4.-validate-events-and-identify-the-causes}

Select any events with PM10 or PM2.5 greater than 95 percentile.
Manually validate events using online newspaper archives, government and
research agency records, satellite imagery and other sources (such as a
Dust Storm database). Enter the information for each event into the
custom built data entry forms. For any events with references for
multiple types of source, assess the liklihood of any single source
being the dominant source. Double check any remaining 99th percentile
dates with no references.

\section{Availability and
requirements}\label{availability-and-requirements}

Lists the following:

\begin{itemize}
\itemsep1pt\parskip0pt\parsep0pt
\item
  Project name: BiosmokeValidatedEvents
\item
  Project home page:
  \url{http://swish-climate-impact-assessment.github.io/BiosmokeValidatedEvents/}
\item
  Operating system(s): R package is platform independent. Data Entry
  forms are Microsoft Windows.
\item
  Programming language: R and SQL
\item
  Other requirements: PostgreSQL (PostGIS is desirable)
\item
  License: CC BY 4.0
\item
  Any restrictions to use: amendments of errors of ommision or
  commission are invited but will be vetted before insertion into the
  master database.
\end{itemize}

\subsection{Availability of supporting
data}\label{availability-of-supporting-data}

BMC Research Notes encourages authors to deposit the data set(s)
supporting the results reported in submitted manuscripts in a
publicly-accessible data repository, when it is not possible to publish
them as additional files. This section should only be included when
supporting data are available and must include the name of the
repository and the permanent identifier or accession number and
persistent hyperlink(s) for the data set(s). The following format is
required:

``The data set(s) supporting the results of this article is(are)
available in the {[}repository name{]} repository, {[}unique persistent
identifier and hyperlink to dataset(s) in \url{http://} format{]}.''

Where all supporting data are included in the article or additional
files the following format is required:

``The data set(s) supporting the results of this article is(are)
included within the article (and its additional file(s))''

We also recommend that the data set(s) be cited, where appropriate in
the manuscript, and included in the reference list.

A list of available scientific research data repositories can be found
here. A list of all BioMed Central journals that require or encourage
this section to be included in research articles can be found here.

\textbf{References}

Atkinson, R.W., Anderson, R.H., Sunyer, J., Ayres, J., Baccini, M.,
Vonk, J.M., Boumghar, A., Forastiere, F., Forsberg, B., Touloumi, G.,
Schwartz, J. \& Katsouyanni, K. (2001). Acute Effects of Particulate Air
Pollution on Respiratory Admissions. \emph{American Journal of
Respiratory and Critical Care Medicine}, \textbf{164}, 1860--1866.

Johnston, F., Hanigan, I., Henderson, S., Morgan, G. \& Bowman, D.
(2011a). Extreme air pollution events from bushfires and dust storms and
their association with mortality in Sydney, Australia 1994-2007.
\emph{Environmental Research}, \textbf{111}, 811--816.

Johnston, F.H., Hanigan, I.C., Henderson, S.B., Morgan, G.G., Portner,
T., Williamson, G.J. \& Bowman, D.M.J.S. (2011b). Creating an integrated
historical record of extreme particulate air pollution events in
Australian cities from 1994 to 2007. \emph{Journal of the Air \& Waste
Management Association}, \textbf{61}, 390--398.

Martin, K.L., Hanigan, I.C., Morgan, G.G., Henderson, S.B. \& Johnston,
F.H. (2013). Air pollution from bushfires and their association with
hospital admissions in Sydney, Newcastle and Wollongong, Australia
1994-2007. \emph{Australian and New Zealand Journal of Public Health},
\textbf{37}, 238--243.


